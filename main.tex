% Magic Comments
% !TeX spellcheck = en_GB
% !TeX encoding = utf8
% !TeX program = pdflatex

\documentclass[12pt]{amsart}

\title{Mathematical Finance in discrete time}
\author{Elias Mindlberger}
\date{Winter Semester 2025/26}

\usepackage{xparse}
\usepackage{amssymb}
\usepackage{eucal}

%%%%%%%%%%%%%%%%%%%%%%%%%%%
%%%% Standard Notation %%%%
%%%%%%%%%%%%%%%%%%%%%%%%%%%

\let\oldepsilon\epsilon
\let\epsilon\varepsilon

\let\oldphi\phi
\let\phi\varphi

\newcommand{\cF}{\mathcal{F}}
\newcommand{\cA}{\mathcal{A}}
\renewcommand{\P}{\mathbb{P}}

%%%%%%%%%%%%% Some Glue Notation
\DeclareMathOperator*{\defn}{\stackrel{\operatorname{def}}{=}}
\DeclareMathOperator*{\interior}{int}
\DeclareMathOperator*{\cl}{cl}

\newcommand{\asconv}{\stackrel{\operatorname{a.s.}}{\longrightarrow}}

\newcommand{\psubsp}{\triangleleft}
\newcommand{\subsp}{\trianglelefteq}
\newcommand{\set}[1]{\left\{ #1 \right\}}

%%%%%%%%%%%%% Commonly encountered sets
\newcommand{\R}{\mathbb{R}}
\newcommand{\N}{\mathbb{N}}

\RenewDocumentCommand{\S}{o}{
    \mathbf{S}
    \IfNoValueTF{#1}{}{^{#1}}
}

\NewDocumentCommand{\B}{o o}{
    B
    \IfNoValueTF{#2}{}{^{#2}}
    \IfNoValueTF{#1}{}{_{#1}}
}

\newcommand{\Lsym}{L}
\RenewDocumentCommand{\L}{o o}{
    \Lsym
    \IfNoValueTF{#1}{}{^{#1}}
    \IfNoValueTF{#2}{}{\left( #2 \right)}
}
\newcommand{\Lp}{\L[p]}
\newcommand{\Lq}{\L[q]}

\RenewDocumentCommand{\l}{o o}{
    \ell
    \IfNoValueTF{#1}{}{^{#1}
    \IfNoValueTF{#2}{}{_{#2}}}
}

\newcommand{\llinfty}{\l[\infty]}

\DeclareMathOperator*{\spn}{span}
\DeclareMathOperator*{\ran}{ran}

%%%%%%%%%%%%% Common Functionals, Measures, Operators
\DeclareMathOperator{\Exp}{\mathbb{E}}

\DeclareMathOperator{\Id}{Id}

\NewDocumentCommand{\T}{o}{
    \IfNoValueTF{#1}{\cdot^{\top}}{{#1}^{\top}}
}

\NewDocumentCommand{\inv}{o}{
    \IfNoValueTF{#1}{\cdot^{-1}}{{#1}^{-1}}
}

\newcommand{\At}{\T[A]}
\newcommand{\Ainv}{\inv[A]}

\renewcommand{\d}{\mathrm{d}}
\NewDocumentCommand{\intt}{o o o o}{
    \int
    \IfNoValueTF{#1}{}{_{#1}}
    \IfNoValueTF{#2}{}{^{#2}}
    \IfNoValueTF{#3}{}{#3}\, 
    \IfNoValueTF{#4}{}{\d #4}
}

\NewDocumentCommand{\abs}{o}{
    \left\lvert
        \IfNoValueTF{#1}{\cdot}{#1}
    \right\rvert
}

\NewDocumentCommand{\norm}{o o}{
    \left\lVert
        \IfNoValueTF{#1}{\cdot}{#1}
    \right\rVert
    \IfNoValueTF{#2}{}{\IfBlankTF{#2}{}{_{#2}}}
}

\NewDocumentCommand{\inn}{o o o}{
    \left(
    \IfNoValueTF{#1}{\cdot}{#1}
    \mid
    \IfNoValueTF{#2}{\cdot}{#2}
    \right)
    \IfNoValueTF{#3}{}{_{#3}}
}

\DeclareMathOperator{\Prsym}{\mathbb{P}}
\RenewDocumentCommand{\Pr}{o}{
    \Prsym
    \IfNoValueTF{#1}{}{\left\{{#1}\right\}}
}


\newtheorem{theorem}{Theorem}
\newtheorem{corollary}[theorem]{Corollary}
\newtheorem{lemma}[theorem]{Lemma}
\newtheorem{exercise}[theorem]{Exercise}

\newcommand*{\from}{\colon}
\usepackage{tcolorbox}
\usepackage{hyperref}
\usepackage{geometry}
\usepackage{tikz}
\usepackage{pgfplots}
\usepackage{booktabs}
\usepackage{amssymb}
\usepackage{xparse}
\usepackage{eucal}
\usepackage{dsfont}
\usepackage{enumitem}


\pgfplotsset{compat=1.18}
\tcbuselibrary{theorems}

\newtcolorbox[auto counter, number within=section]{definition}[1][]{%
  colback=blue!5,
  colframe=blue!40!black,
  fonttitle=\bfseries,
  title={Definition~\thetcbcounter\ifstrempty{#1}{}{: #1}},
}

\newtcolorbox[auto counter, number within=section]{lemma}[1][]{%
  colback=green!5,
  colframe=green!40!black,
  fonttitle=\bfseries,
  title={Lemma~\thetcbcounter\ifstrempty{#1}{}{: #1}},
}

\newtcolorbox[auto counter, number within=section]{theorem}[1][]{%
  colback=red!5,
  colframe=red!40!black,
  fonttitle=\bfseries,
  title={Theorem~\thetcbcounter\ifstrempty{#1}{}{: #1}},
}

\newtheorem{corollary}[subsection]{Corollary}
\newtheorem{exercise}[subsection]{Exercise}

\newtheorem*{remark}{Remark}

\begin{document}

\maketitle

\tableofcontents
\bigskip

\section{Introduction and first example}

We start by introducing the first important notions of option pricing and first examples.

\subsection{Notions for option pricing theory}

An \textbf{option} is a contract beween two parties (a buyer and a seller). The buyer pays an option price \emph{today} (i.e.\ at \(t=0\)) to the seller and in return obtains the right/\emph{option} \underline{but not the obligation} to buy a stock for conditions that are fixed \emph{today} at a fixed point in time (\emph{the maturity of the option}).
The option to buy is thereby crucial: the buyer can just refrain from buying the stock if te market conditions are disadvantageous.

We only consider options whose \emph{underlyings (german: Basistitel)} are stocks.\newpage

We distinguish the following:

\begin{tikzpicture}[>=stealth, thick]
    % Arrows for call and put
    \draw[->] (-3,0) -- (3,2);
    \draw[->] (-3,0) -- (3,-2);

    % Labels
    \node at (4.5,2) {\textbf{Call option}};
    \node at (4.5,-2) {\textbf{Put option}};

    % Text below arrows
    \node[align=left] at (6,1) {Buyer has the right to \underline{buy} \\ the stock for the \emph{strike price} \(K\).};
    \node[align=left] at (6,-3) {Buyer has the right to \underline{sell} \\ the stock for the \emph{strike price} \(K\).};

    % Arrows for call and put
    \draw[->] (-3,-7) -- (3,-5);
    \draw[->] (-3,-7) -- (3,-9);

    % Labels
    \node at (5,-5) {\textbf{European option}};
    \node at (5,-9) {\textbf{American option}};

    % Text below arrows
    \node[align=left] at (6.5,-5.7) {Can only be exercised at maturity \(T\).};
    \node[align=left] at (6.5,-9.7) {Can be exercised at \(t\) s.t.\ \(0 \leq t \leq T\).};
\end{tikzpicture}

We call the above \textbf{standard} or \textbf{vanilla} options. Other options are called \textbf{exotic}. Such include \emph{Asian options} (which operate on the mean--value of an asset), \emph{barrier options}, \emph{options on voatility} and so forth. For this course we assume that options have a price at any point in time \(0 \leq t \leq T\). Advantages of options include
\begin{itemize}
    \item The loss is limited: worst case is the loss of the option price,
    \item leverage effect (in a positive market development),
    \item can be used to \emph{hedge} against decreasing stock prices via put options (can be sold for a fixed pre--determined price).
\end{itemize}

\subsection{A first example}

Let us consider a call option in a two--period market model. We set
\begin{itemize}
    \item \(t=0\): the current time or today,
    \item \(T>0\): the maturity or exercise date of the option,
    \item \(S_T\): random price of the underlying stock at time \(T\),
    \item \(S_0\): known price of the underlying stock at time \(t=0\),
    \item \(K\): the strike price.
\end{itemize}

\begin{remark}
    No physical transaction of the stock at time \(T\) happens. Only the profit is paid out accordingly.
\end{remark}
Clearly, the \textbf{payoff} for a call option at time \(T\) is \[
    H = \max \set{S_T-K, 0} = \left(S_T-K\right)^+.
\]
\(H\) itself is a random variable since \(S_T\) is assumed to be random.


\begin{center}
    \begin{tikzpicture}
        \begin{axis}[
            axis lines=middle,
            samples=200,
            domain=-1:4,
            ymin=-0.5, ymax=3,
            xmin=-1, xmax=4,
            axis line style={->, thick},
            xtick=\empty,
            ytick=\empty,
            xlabel={\(S_T\)},
            ylabel={Payoff},
            width=8cm,
            height=6cm
        ]

        % Dashed part for x < K
        \addplot[blue, ultra thick, dashed, domain=-1:1.5] {0};

        % Solid part for x >= K
        \addplot[blue, ultra thick, domain=1.5:4] {x-1.5};

        % Mark K
        \draw[dashed, gray] (axis cs:1.5,0) -- (axis cs:1.5,1.5);
        \filldraw[black] (axis cs:1.5,0) circle (1.5pt) node[below] {\(K\)};

        % Function label
        \node[blue, above right] at (axis cs:3,1) {\(H\)};

        \end{axis}
    \end{tikzpicture}

    \vspace{0.5em}
    \small\textbf{Figure:} Visualisation of the payoff \(H\) dependent on the strike \(K\).
\end{center}

Take \(S_0=100\) and assume the stock can attain two prices at \(T\). Either it attains \(150\) in the case of \(\omega_1\) or it attains \(90\) in the case of \(\omega_2\), i.e.\ \(S_T(\omega_1) = 150\) and \(S_T(\omega_2) = 90\) with probabilities \(p\) and \(1-p\) respectively. We let \(130\) be the strike \(K\). The payoff is obviously \(20\) and \(0\) respectively. We assume that there is an additional, riskless investment opportunity with interest rate \(r=0\).

\textbf{Question.} What is the fair price \(\pi(H)\) of the above option at \(t=0\)?

\textbf{Idea: No--arbitrage--principle.}
There is no arbitrage (i.e.\ no riskless profit) in the financial market.

The payoff \(H\) is replicated by other assets (in this case stocks and riskless investments).

The initial capital that is needed in order to replicate \(H\) corresponds to the price of the option. Otherwise, there is arbitrage.

We now want to find a \textbf{trading strategy} \((\alpha, \beta) \in \R^2\) with
\begin{itemize}
    \item \(\alpha\), the number of stocks that we buy at \(t=0\),
    \item \(\beta\), the investment in the riskless asset (RA).
\end{itemize}

The value of the portfolio at \(t=0\) is \(V_0(\alpha, \beta) = \beta \cdot 1 + \alpha S_0\) (the RA is normalised) and the value of the portfolio at \(t=T\) is \(V_T(\alpha, \beta) = \beta \cdot 1 + \alpha S_T\) (recall \(r=0\)). We want to \emph{hedge} or \emph{replicate} the payoff, i.e.\ choose \((\alpha, \beta)\) such that \(V_T(\alpha, \beta) = H\). This means \(\beta+\alpha S_T(\omega) = H(\omega)\) for any \(\omega \in \set{\omega_1, \omega_2}\). Hence, we get a system of linear equations.

\begin{align*}
    \beta + 150\alpha &= 20\\
    \beta + 90 \alpha &= -0.
\end{align*}
Solving this for \(\alpha, \beta\) yields \(\alpha=1/3\) and \(\beta=-30\). Hence \[
    V_0(\alpha, \beta) = -30 + 100/3 = 10/3 = \pi(H).
\]
The hedging strategy that we get is to borrow \(30\) today and buy \(1/3\) stocks. The overall investment is \(10/3\). Two scenarios can happen at \(t=T\):
\begin{enumerate}
    \item \(S_T=150\). Selling the stock yields \(1/3 \cdot 150 = 50\). We repay our debt of \(30\) and yield \(20\) as profit.
    \item \(S_T=90\). Selling the stock yields \(1/3 \cdot 90 = 30\). We repay the debt and nothing happens.
\end{enumerate}
As we see, we hedged \(H\) perfectly. Now assume that we \underline{sell} the call option at a price of \(10/3\) and invest. Then, the following can happen.

\begin{enumerate}
    \item \(S_T=150\). The buyer will exercise the call, he or she will buy the stock for \(130\) from us,
    \item we have to buy the stock for \(150\), we lose \(20\),
    \item but we obtain \(20\) as by hedging in (1).
\end{enumerate}

\begin{enumerate}
    \item \(S_T=90\). Goes analogously.
\end{enumerate}

Using the no--arbitrage--principle, the fair price of the option is \(\pi(H)=10/3\). For any other price, a riskless profit would be possible. Assume \(\pi(H) > 10/3\). Then we get the following table:

\begin{table}[h!]
    \centering
        \begin{tabular}{ll}
            \toprule
            \textbf{Action at \(t=0\)} & \textbf{Payoff at \(t=T\)} \\
            \midrule
            Sell \(\pi(H)\) & \(-(S_T-K)^+\)\\
            Borrow \(30\) RA & \(-30\)\\
            Buy \(S_0/3\) & \(S_T/3\)\\
            Profit: \(\pi(H)+30-S_0/3=\pi(H)-10/3>0\) & \(\Sigma=0\) \\
            \bottomrule
        \end{tabular}
    \caption{Unfair pricing shows the possibility of riskless profit.}
\end{table}

Assume now \(\pi(H) < 10/3\). The table goes analogous to the above.

\begin{remark}
    Notice that the fair price \(\pi(H)\) in this example is \textbf{independent} of the subjective probability \(p\)! In more sophisticated models, we will also adapt to this.
\end{remark}

\subsection{Market assumptions}

In this section, we used a perfect market model.
\begin{itemize}
    \item The markets are frictionless, i.e.\ there are no taxes on profits and no transaction costs for reallocating portfolios,
    \item short--selling is allowed at all times and arbitrary shares can be bought and sold,
    \item interest rates for borrowing and lending are te same as well as investments in RA,
    \item there are no dividend payments,
    \item all market participants are rational and maximise their utility.
\end{itemize}

In a perfect market there is no arbitrage. Supply and demand offset each other perfectly and there are unique prices.

\section{Financial markets and the finite state space}

Our goal here is to introduce general financial markets. Finite means that there is a finite number of market states and a finite number of trading times \(\set{0, \dots, T}\). We describe trading strategies, arbitrage strategies and options formally.

\subsection{Financial market}

We take \((\Omega, \cF, (\cF_t)_{t}, \P)\) to be a filtered probability space. I.e.\
\begin{itemize}
    \item \(\Omega\) is a finite state space of elementary events,
    \item \(\cF\) is the power set of \(\Omega\) and acts as the \(\sigma\)-algebra,
    \item \((\cF_t)_t\) is a filtration,
    \item \(\P: \cF \to [0, 1]\) is a probability measure.
\end{itemize}

\begin{remark}
    We remark that for a probability space \((\Omega, \cF, \P)\) and an \underline{ordered} index set \(I\), a family of \(\cF\)-sub-\(\sigma\)-algebras \((\cF_i)_{i \in I}\) is a filtration if and only if for any \(t, s \in I: t \leq s \implies \cF_t \subseteq \cF_s\). Since our index sets \(I\) are always finite with maximal element \(T\), we make the additional presumption that \(\cF = \cF_T\).
\end{remark}

\section{The Cox--Ross--Rubinstein model}
\section{Absence of arbitrage and equivalent martingale measures}
\section{Completeness and equivalent martingale measures}
\section{Risk--neutral pricing of contingent claims}
\section{American options}
\section{Portfolio optimization}

\end{document}